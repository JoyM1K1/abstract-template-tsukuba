% 独自のコマンド

% ■ 謝辞
%  \begin{acknowledgment} 〜 \end{acknowledgment}

% ■ 文献リスト
%  \begin{bib}[100] 〜 \end{bib}


\newif\ifjapanese

\japanesetrue  % 論文全体を日本語で書く(英語で書くならコメントアウト)

\ifjapanese
  \documentclass[a4j,twocolumn,11pt,dvipdfmx]{ujarticle} % 両面印刷の場合。余白を綴じ側に作って右起こし。
  %\documentclass[a4j,11pt,dvipdfmx]{ujreport}                  % 片面印刷の場合。
\else
  \documentclass[a4paper,11pt]{report}
  \newcommand{\acknowledgmentname}{Acknowledgment}
\fi

% packages
\usepackage{thesis}
\usepackage{ascmac}
\usepackage{amsmath, amsfonts, mathrsfs} % 筆記体:mathcal, 花文字:mathscr
\usepackage{mathtools}
\usepackage{bm}
\usepackage{graphicx}
\usepackage{multirow}
\usepackage{url}
\usepackage{booktabs}
\usepackage{siunitx}
\usepackage[top=2.5cm,bottom=2cm,left=2cm,right=2cm]{geometry}
\usepackage{macro}

% \bibliographystyle{jplain}

% 画像のpath
\graphicspath{{../graphics/}}

\newcommand{\kintou}[2]{%
\leavevmode
\hbox to #1{%
\kanjiskip=0pt plus 1fill minus 1fill
\xkanjiskip=\kanjiskip
#2}}

\pagestyle{empty}

\begin{document}

\twocolumn[
  \begin{center}
    {\large 数理物質研究科 修士論文概要}
  \end{center}
  \begin{flushright}
    \begin{minipage}{0.35\textwidth}
      \begin{flushleft}
        \kintou{4zw}{専攻名}  \quad 専攻名\\
        \kintou{4zw}{学籍番号} \quad 学籍番号\\
        \kintou{4zw}{学生氏名} \quad 学生氏名\\
        \kintou{4zw}{学位名}  \quad 修士(理学)\\
        \kintou{4zw}{指導教員} \quad 教員氏名 印
      \end{flushleft}
    \end{minipage}
  \end{flushright}
  \begin{center}
    {\large 修士論文題目 題目題目題目題目題目題目題目}
  \end{center}
  \quad
  ここに概要.ここに概要.ここに概要.ここに概要.ここに概要.ここに概要.ここに概要.ここに概要.ここに概要.ここに概要.ここに概要.ここに概要.ここに概要.ここに概要.ここに概要.ここに概要.
  \\
]

ここに概要.ここに概要.ここに概要.ここに概要.ここに概要.ここに概要.ここに概要.ここに概要.ここに概要.ここに概要.ここに概要.ここに概要.ここに概要.ここに概要.ここに概要.ここに概要.

% 参考文献がある場合は記述
% \bibliographystyle{junsrt}
% \bibliography{main}
\end{document}
